\begin{conclusions}
    Luego de finalizar este trabajo podemos afirmar que se ha cumplido con los requierimientos englobados por el problema inicial, planteados
    en el epígrade \ref{section:formulacionProblema}. Se desarrolló la aplicación móvil Inventario que permite acceder a dos funcionalidades principales que son: permitir
    a los usuarios finales, los cuales serán encuestadores de profesión, rellenar información sobre los predios que le fueron asignados, y
    visualizar un mapa totalmente sin conexión a Internet como capa base con varias capas de delimitaciones, que permitan observar mejor el área a encuestar.\\
    Además se debe señalar que se realizó un gran bogeo investigativo por el estado del arte, constatado en el Capítulo \ref{chapter:state-of-the-art},
    ayudando a posteriores trabajos en la rama de Sistemas de Información Geografica a tener una base de investigación un poco más amplia y de actualidad. Además, no solo se lograron
    los objetivos generales y específicos definidos en el epígrade \ref{section:objetivosGeneralesYEspecificos}, sino que el trayecto por el desarrollo de la aplicación superó con creces las espectativas
    gracias al aprendizaje adquirido sobre Flutter y también al trabajo de estudio, diseño e integración de la arquitectura de software MVVM sobre el proyecto(Ver Capítulo \ref{chapter:proposal})\\
    Con el desarrollo de esta aplicación móvil se puso en evidencia todo el conocimiento
    adquirido en la carrera y la habilidad investigativa para resolver problemáticas propias
    del desarrollo de software.

\end{conclusions}
