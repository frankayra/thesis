\begin{opinion}
    El trabajo presentado como tesis de culminación de estudios de licenciatura por la estudiante Francisco Octavio Ayra Cáceres  constituye un ejemplo de la aplicación de conocimientos y habilidades adquiridas durante sus años de estudio como Programación, Bases de Datos e Ingeniería  de Software, combinados con la utilización de su capacidad de investigación para la generación de ideas, asimilación de tecnologías y creación de soluciones al problema planteado.
    Para cumplir con los objetivos propuestos, realizó una ardua labor investigativa en temas totalmente nuevos para él, relacionados con las tecnologías presentes en los dispositivos móviles actuales y con la amplia temática de los Sistemas de Información Geográfica  las redes topológicas.
    Este esfuerzo se materializa en lo expuesto en los Capítulos 1 y 2, donde recoge el estado del arte relacionado con los objetivos de este trabajo de diploma, tanto desde el punto de vista teórico-conceptual como tecnológico y en el Capítulo 3, propone e implementa  una solución teórico- computacional  para solucionar su problemática haciendo uso de la tecnología asimilada. Independencia, dedicación, perseverancia, profesionalidad, capacidad e iniciativa son algunos de los calificativos que merece la labor de. Además, demostró haber adquirido las bases de la metodología de la investigación científica, resultado que puede ser valorado en el presente diploma. Francisco
    Como tutora estoy muy satisfecha con el trabajo de Francisco y considero que desde el punto de vista académico, satisface los requerimientos de una tesis de licenciatura y cumple con los objetivos que nos trazamos para la misma al inicio de esta etapa.
    Por todo lo anterior propongo al tribunal que se le otorgue la calificación de Excelente (5).
    La Habana, julio del 2025
    MSc. Joanna Campbell Amos

\end{opinion}