\begin{resumen}

	En este trabajo de diploma se presenta una propuesta de solución para el desarrollo
	de una herramienta para el sistema operativo Android llamada “Inventario” con el
	objetivo de que agentes encuestadores puedan recoger información referente a
	determinados terrenos o edificaciones, teniendo de antemano la geolocalización de
	los mismos, entre otras cosas como la numeración de las parcelas y la delimitación
	entre las mismas. Todo esto se pretende llevarlo a cabo a partir de una
	cartografía(gestión de mapas) desconectada de internet, manteniendo una base de
	datos local. Además se mantiene un desacoplamiento de la base de datos que
	permita la posterior unificación de los datos por separado de cada encuestador en
	una misma base de datos coherente y centralizada mediante una herramienta de
	software externo. En el proceso de investigación para el desarrollo de la propuesta,
	se analizaron distintas herramientas con funcionalidades similares a las que se
	necesitan, para tomar experiencias e investigar sobre el estado del arte en el ámbito
	de los Sistemas de Información Geográfica(SIG)\cite{SIG}. También se realizó un estudio
	de diferentes plataformas de desarrollo disponibles para el manejo de mapas sin
	conexión, teniendo en cuenta sus ventajas y desventajas. A raíz de toda esta
	investigación se consolidó una propuesta de solución que cubre las funcionalidades
	deseadas para Inventario, haciendo uso de Flutter\cite{flutterOficial}, una librería multiplataforma
	del lenguaje Dart\cite{dartOficial}, utilizado principalmente para el desarrollo de aplicaciones móviles,
	empleando la sublibrería flutter\_map\cite{flutterMap} para el manejo de mapas offline y una Arquitectura de Modelo-Vista-Modelo de Vista\cite{MVVM}
	como infraestructura para la aplicación.


\end{resumen}

\begin{abstract}
	This diploma work presents a proposed solution for the development of a tool for the Android operating system called “Inventory”
	with the aim that surveyors agents can collect information about certain land or buildings, having in advance the geolocation
	of the same, among other things such as the numbering of the parcels and the delimitation between them.
	All this is intended to be carried out from a cartography (map management) disconnected from the internet, maintaining a local database.
	In addition, a decoupling of the database is maintained to allow the subsequent unification of the separate data of each surveyor in the
	same coherent and centralized database by means of an external software tool. In the research process for the development of the proposal,
	different tools with similar functionalities to those needed were analyzed, in order to gain experience and investigate the
	state of the art in the field of Geographic Information Systems (GIS)\cite{SIG}. A study of different development platforms available for offline map management was
	also carried out, taking into account their advantages and disadvantages. As a result of all this research, a solution proposal was consolidated
	that covers the desired functionalities for Inventory, making use of Flutter\cite{flutterOficial}, a multiplatform library of the Dart language\cite{dartOficial},
	mainly used for the development of mobile applications, using the flutter\_map sublibrary\cite{flutterMap} for offline map management and a Model-View-View Model Architecture\cite{MVVM}
	as infrastructure for the application.

\end{abstract}